\documentclass[aps, pra, a4paper, longbibliography]{revtex4}
% [VB] needs to be changed to revtex4-1 but my current latex setup is
% ancient so we'll need to use this for a while

\usepackage[utf8]{inputenc}
\usepackage[T1]{fontenc}
\usepackage[australian]{babel}
\usepackage{graphicx, hyperref, amsmath, amssymb, verbatim}

\newcommand{\I}{I}
\newcommand{\be}{\begin{equation}}
\newcommand{\ee}{\end{equation}}
\newcommand{\eq}{\Leftrightarrow}

\newcommand{\ket}[1]{\left| #1 \right \rangle}
\newcommand{\bra}[1]{\left \langle #1 \right|}
\newcommand{\braket}[2]{\left \langle #1 | #2 \right \rangle}
\newcommand{\ketbra}[2]{\left| #1 \right \rangle \left \langle #2 \right|}
\newcommand{\comm}[2]{\left[ #1, #2 \right]}
\newcommand{\inprod}[2]{\left\langle #1, #2 \right\rangle}

\newcommand{\hilb}[1]{\mathcal{#1}}

\DeclareMathOperator{\tr}{tr}
\DeclareMathOperator{\re}{Re}
\DeclareMathOperator{\cvec}{vec}

\newcommand{\dd}[2]{\frac{\partial #1}{\partial #2}}


% TODO Dynamo vs DYNAMO
\begin{document}
\title{DYNAMO manual}
\date{\today}

\author{Ville Bergholm}
\email{ville.bergholm@iki.fi}
\affiliation{Department of Chemistry, Technische Universität München, Germany}

% \pacs{}
% \keywords{tensor network, invariant} not needed as they are embedded into the PDF.  

\begin{abstract}
Dynamo...
\end{abstract}
\maketitle



\tableofcontents


\section{Introduction}

Dynamo~\cite{machnes_2011},



\section{Basics}
% TODO the word system is used for both S alone and the whole thing
The Dynamo package is designed to handle four kinds of quantum
mechanical systems.
\begin{itemize}
\item[S:] closed system (unitary)
\item[SB:] open system coupled to a Markovian bath (Lindblad)
\item[SE:] system with environment (unitary)
\item[SEB:] system and environment coupled to a Markovian bath (Lindblad)
\end{itemize}
In the simplest case we have a closed system~$S$ with unitary
propagation generated by the system Hamiltonian.
$S$~may also be open, coupled to a Markovian bath~$B$, in which
case the evolution will be generated by a Liouvillian of the Lindblad form.
In addition to~$S$ we may also have an environment~$E$ coupled to
it. The evolution of the compound system $S+E$ is treated
fully quantum mechanically enabling non-Markovian recurrence effects,
but in the end we are only interested in the state of~$S$. In this
case, too, $S+E$ may be closed or coupled to a Markovian bath~$B$.

All these cases can be described by a linear master equation of the following form:
\be
\label{eq:master}
\dot{X} = -\underbrace{(A(t) +\sum_{c} u_c(t) B_c)}_{G(t)} X(t) = -G(t) X(t),
\ee
where $u_c$~are the (scalar) \emph{control fields}, $B_c$~the
corresponding \emph{control generators},
$A$~the \emph{drift} generator, $X$~the ``system'' (vector or operator), and $G$~the total generator.
The minus sign in front of~$G$ is a convention.

$X$ in Eq.~\eqref{eq:master} can refer to either a Hilbert space
vector (ket) or operator, or a Liouville space vector
%(Hilbert space state operator)
or operator.
We use the $\cvec$-representation of state
operators, explained in Appendix~\ref{sec:vec},
as the mapping between the Hilbert and Liouville spaces.


% sum over control fields: c
% sum over time slices: k
Assume that the controls are
piecewise constant in time: $u_{c, k}$, with $n$~time slices in total,
and that the duration of the $k$th slice is~$\tau_k$, giving
\be
t_k = t_0 + \sum_{j=1}^{k} \tau_j.
\ee
For each time slice we now obtain the propagator
\be
P_k := \exp(-G_k \tau_k),
\ee
with
\be
X_k := X(t_k) = \prod_{j=1}^{k} P_j X(t_0).
\ee
The propagators, like the generators, always act on~$X$ by multiplication
from the left.

Derivatives:
\begin{align}
\dd{P_k}{\tau_k}  &= -G_k P_k = -P_k G_k,\\
\dd{P_k}{u_{c,k}}
&=
\sum_{j=0}^{\infty} \frac{(-\tau_k)^j}{j!}
\sum_{q=0}^{j-1}
G_k^{q} \dd{G_k}{u_{c,k}} G_k^{j-q-1}
=
\sum_{j=0}^{\infty} \frac{(-\tau_k)^j}{j!}
\sum_{q=0}^{j-1}
G_k^{q} B_c G_k^{j-q-1}
\approx -\tau_k B_c P_k.
\end{align}
The approximation is exact if $\comm{B_c}{G_k} = 0$.


System and adjoint system propagators:
\begin{align}
U_k &:= P_k \cdots P_1,\\      % U_k X_0 = X_k
\Lambda_k &:= P_n \cdots P_{k+1}.
\end{align}
NOTE: currently the code uses a different definition:
$U_{k+1} = P_k \cdots P_1 X_0 = X_k$ and
$\Lambda_k = X_f^\dagger P_n \cdots P_k$ (MATLAB indexing limitations...).

\begin{table}[h]
\[
\begin{array}{c}
\begin{array}{@{t_0}p{1.9em}@{t_1}p{1.8em}@{t_2}p{1.4em}@{t_{k-1}}p{1.1em}@{t_{k}}p{1.1em}@{t_{k+1}}p{0.8em}@{t_{n-2}}p{0.8em}@{t_{n-1}}p{1.3em}@{t_n}}
& & & & & & &
\end{array}\\
\begin{array}{|p{2em}|p{2em}|p{2em}|p{2em}|p{2em}|p{2em}|p{2em}|p{2em}|}
 $\tau_1$ & $\tau_2$ & & $\tau_k$ & $\tau_{k+1}$ & & $\tau_{n-1}$ & $\tau_n$ \\
% $u_{c,1}$ & $u_{c,2}$ & & $u_{c,k}$ & $u_{c,k+1}$ & & $u_{c,n-1}$ & $u_{c,n}$ \\
 $P_1$ & $P_2$ & $\cdots$ & $P_k$ & $P_{k+1}$ & $\cdots$ & $P_{n-1}$ & $P_n$ \\
\cline{1-4}
& & & $U_{k}$ & $\Lambda_{k}$ & & & \\
\cline{5-8}
\end{array}
\end{array}
\]
\caption{Time slices and operators related to them.
$t_k = t_0 + \sum_{j=1}^{k} \tau_j$.
The total forward and backward
propagators to the point $t_k$ are defined as
$U_k = P_k \cdots P_1$ and
$\Lambda_k = P_{n} \cdots P_{k+1}$.}
\end{table}








\section{Optimization}

In what follows, all quantum states are assumed to be normalized to
unity. $N = \dim \hilb{H}$~is the dimension of the Hilbert space of the system.
The results of this section are summarized in Table~\ref{table:tasks}


Within each type of system we have several possible optimization
tasks.
We usually wish to minimize an operator/vector distance, measured using the Frobenius norm:
\be
d^2(A, B) = |A-B|^2
= \inprod{A-B}{A-B}
%=\tr((A-B)^\dagger (A-B))
%= |A|^2 +|B|^2 -\tr(A^\dagger B) -\tr(B^\dagger A)
%= |A|^2 +|B|^2 -2 \re \tr(A^\dagger B).
= |A|^2 +|B|^2 -2 \re \inprod{A}{B}.
\ee
Dividing this expression with the target norm~$|A|^2$ (assumed fixed,
known), we obtain the normalized distance measure
\be
\label{eq:df}
D(A,B)
:= \frac{d^2(A, B)}{|A|^2}
= 1 +\frac{|B|^2}{|A|^2} -2 f(A, B),
\ee
where
\be
f(A, B) := \frac{1}{|A|^2} \re \tr(A^\dagger B)
\ee
is the normalized \emph{fidelity}\footnote{
There is another widely used quantity called fidelity in quantum information science which is different from the present one.}.
% NOTE fidelity is the real part of an inner product, one can use Cauchy-Schwarz inequality...
Clearly $f(A, A) = 1$, and
the triangle inequality $|A-B| \le |A|+|B|$ yields
\be
\left(1 -\frac{|B|}{|A|} \right)^2 \le D(A, B) \le \left(1 +\frac{|B|}{|A|} \right)^2
\qquad \text{and} \qquad
|f(A, B)|
\le \frac{|B|}{|A|}
%\le \frac{1}{2} \left(1 +\frac{|B|^2}{|A|^2} \right)
\ee
since $f(A, -B) = -f(A, B)$.


(X1): Is minimum distance equivalent to maximum fidelity?
If this holds, we may optimize fidelity instead of distance.
If $|A|^2$ and $|B|^2$ are constant
from Eq.~\eqref{eq:df} we can see that (X1) clearly holds.






\begin{table}
\begin{tabular}{l|c|c|l|l|l|l}
& task & H/L & X & $|X|^2$ & error function & $f_\text{max}$\\
\hline
Closed system S
& $\rho_i \to \rho_f$ & L & $\cvec(\rho)$ & $P(\rho)$
& $E_\text{real}$ & $\sqrt{\frac{P(\rho_i)}{P(\rho_f)}}$\\
& $\ket{\psi_i} \to \ket{\psi_f}$ & H & $\ket{\psi}$ & 1 &
$E_\text{abs}$ & 1\\
& $V_i \to V_f$ & H & $V$ & $N$ & $E_\text{abs}$ & 1\\
\hline
Open system SB
& $\rho_i \to \rho_f$ & L & $\cvec(\rho)$ & $P(\rho)$ & $E_\text{open}$\\
& $F_i \to F_f$ & L & $F$ &  & $E_\text{open}$\\
\end{tabular}
\caption{Summary of the optimization tasks.}
\label{table:tasks}
\end{table}


\subsubsection{Error functions}

\begin{align}
E_\text{real}(A, B) &:= f_\text{max} -\frac{1}{|A|^2} \re \tr(A^\dagger B),\\
E_\text{abs}(A, B) &:= f_\text{max} -\frac{1}{|A|^2} |\tr(A^\dagger B)|,\\
E_\text{open}(A, B) &:= 1 +\frac{|B|^2}{|A|^2} -2 \frac{1}{|A|^2} \re \tr(A^\dagger B).
\end{align}


Auxiliary function~$g$:
\be
g(A, B) := \tr(A^\dagger B).
\ee
Gradients:
\begin{align}
\dd{E_\text{real}(A,B)}{u}
&= -\frac{1}{|A|^2} \re \left( \dd{g}{u} \right),\\
\dd{E_\text{abs}(A,B)}{u}
&= -\frac{1}{|A|^2} \re \left(\frac{g^*}{|g|} \dd{g}{u} \right),\\
\dd{E_\text{open}(A, B)}{u}
&= \frac{2}{|A|^2} \re \tr\left((B-A)^\dagger \dd{B}{u}\right).
\end{align}


%The norm derivative gives
%\be
%\dd{|B|^2}{u}
%= \dd{}{u} \tr(B^\dagger B)
%= \tr\left(B^\dagger \dd{B}{u}\right)
%+\tr\left(\dd{B^\dagger}{u} B \right)
%= 2 \re \tr\left(B^\dagger \dd{B}{u}\right).
%\ee



\subsection{Closed system S}

In a closed system, the propagators~$P_k$ are always
unitary. Consequently, the purity of a state is conserved.


\subsubsection{Mixed state transfer $\rho_i \to \rho_f$}
\label{sec:closed-mixed}

To match the form of Eq.~\eqref{eq:master},
the state operators are treated as vectors in Liouville space, $X~=~\cvec(\rho)$.
%\footnote{It will of course turn out that this is equivalent to treating them as operators in Hilbert
%space, $X~=~\rho$ as far as the results are concerned.}
The goal here is to minimize state operator distance
$D(X_f, X_n) = D(\rho_f, \rho_n)$.
$|X|^2$~is equivalent to the purity of the state:
\be
|X|^2
= |\cvec(\rho)|^2
= \tr(\rho^\dagger \rho)
= \tr(\rho^2)
= P(\rho).
\ee
Unitary propagation conserves purity, hence (X1) holds, and we can
simply maximize the fidelity
\be
f(X_f, X_n)
%= \frac{1}{P(\rho_f)} (\re) \tr(X_f^\dagger X_n)
= \frac{1}{P(\rho_f)} (\re) (\tr) \left( X_f^\dagger  P_n \cdots P_1 X_i \right).
\ee
Furthermore, the fidelity is strictly nonnegative since the
state operators are positive:
\be
0 \le f(X_f, X_n) \le \sqrt{\frac{P(\rho_i)}{P(\rho_f)}}.
\ee
If either $\rho_f$ or $\rho_i$ is pure,
$\rho = \ketbra{\psi}{\psi}$,
we have $|\rho|^2 = \braket{\psi}{\psi}^2 = 1$, and
the diagram simplifies by splitting up.


In some cases we are only interested in maximizing the overlap of the state with....




TODO
Alternatively, we can choose $X = \rho$ at the expense of a slightly
more complicated expression for the fidelity:
\begin{comment}
\be
f(X_f, X_n)
= \frac{1}{P(X_f)} (\re) \tr(X_f^\dagger X_n)
= \frac{1}{P(X_f)} (\re) \tr(X_f^\dagger  P_n ... P_1 X_i P_1^\dagger ... P_n^\dagger)
\ee

\be
X_i := \left(\prod_{j=1}^{i} P_j\right) X_i \left(\prod_{j=1}^{i} P_j\right)^\dagger
\ee

\begin{align}
\dd{f(X_f, X_n)}{u(t_j)}
&= \re \left(\dd{g}{u(t_j)} \right)
= \frac{1}{P(X_f)} (\re) \tr \left(X_f \dd{X_n}{u(t_j)}\right)\\
&= \frac{1}{P(X_f)} \left(\tr \left(X_f P_n \cdots \dd{P_j}{u(t_j)} \cdots P_1 X_i P^\dagger_1 \cdots P^\dagger_n\right)
+\tr\left(X_f P_n \cdots P_1 X_i P^\dagger_1 \cdots \dd{P^\dagger(t_j)}{u(t_j)} \cdots P^\dagger_n\right)\right)\\
&= \frac{2}{P(X_f)} \re \tr\left(X_f P_n \cdots \dd{P_j}{u(t_j)} \cdots P_1 X_i P^\dagger_1 \cdots P^\dagger_n\right).
\end{align}
NOTE: last line not in paper!
\end{comment}



\subsubsection{Pure state transfer $\ket{\psi_i} \to \ket{\psi_f}$}
\label{sec:closed-pure}

Using the results from the previous section with both states pure,
$\rho = \ketbra{\psi}{\psi}$, with
$X = \cvec(\rho) = \ket{\psi^*} \otimes \ket{\psi}$,
the fidelity diagram breaks into two pieces and
we obtain
\be
f(X_f, X_n)
= (\re) \left|(\tr) \bra{\psi_f}  P_n \cdots P_1 \ket{\psi_i} \right|^2.
\ee
with $0 \le f(X_f, X_n) \le 1$.
Thus the problem simplifies back into Hilbert space
(albeit with an extra absolute value squared in the expression for the fidelity), and we may equivalently
choose $X = \ket{\psi}$.


%Maximize state overlap
% TODO redefine X, in which space does P operate?


If global phase matters (NOTE: this is unphysical), we may define $X = \ket{\psi}$ and minimize
$D(X_f, X_n)$ directly, and since (X1) holds,
equivalently maximize the corresponding fidelity
%\be
%f(X_f, X_n)
%= \re (\tr) \left(X_f^\dagger P_n \cdots P_1 X_i \right),
%\ee
obeying
$|f(X_f, X_n)| \le 1$.



\subsubsection{Unitary propagator}

In this task we wish to generate a unitary gate~$V_f$ up to global
phase, starting from the identity~$V_i = \I$.
We can get rid of phase by explicitly lifting the problem into
Liouville space (see Eq.~\eqref{eq:L-unitary}),
$X = \hat{V} = V^* \otimes V$,
and then minimize the operator distance~$D(X_f, X_n)$.

Using Eq.~\eqref{eq:hat-product}, the norm squared is 
\be
|\hat{V}|^2
= \tr(\hat{V}^\dagger \hat{V})
= |\tr(V^\dagger V)|^2
= |\tr(\I)|^2
= N^2.
\ee
%\be
%|U|^2 = \tr(U^\dagger U) = \tr(\I) = N.
%\ee
This is constant, so (X1) holds and we may maximize the fidelity instead:
\be
f(X_f, X_n)
= \frac{1}{N^2} \re \tr \left(\hat{V_f}^\dagger \hat{V_n} \right)
%= \frac{1}{N^2} (\re) \left| \tr \left(V_f^\dagger V_n \right) \right|^2
= \frac{1}{N^2} (\re) \left| \tr \left(V_f^\dagger P_n \cdots P_1 V_i \right) \right|^2.
\ee
It is clearly obeys $0 \le f(X_f, X_n) \le 1$.
Much like in
Sec.~\ref{sec:closed-pure},
the problem simplifies back into Hilbert space, and we may equivalently
choose~$X = V$.

If global phase matters (NOTE: unphysical), we may choose~$X = V$ and directly minimize
the operator distance~$D(X_f, X_n)$.
For a unitary operator~$V$ we have $|V|^2 = N$, hence (X1) holds, and
we can equivalently maximize the fidelity
%\be
%f(X_f, X_n)
%= \frac{1}{N} \re \tr \left(X_f^\dagger P_n \cdots P_1 X_i \right)
%\ee
which obeys $|f(X_f, X_n)| \le 1$.



\subsection{Open system with bath S+B}

In this case we have a system~$S$ coupled to a Markovian bath~$B$.
The generators~$G_k$ are now Liouvillians, expressible in terms of
Hamiltonians and Lindblad operators.


\subsubsection{Mixed state transfer $\rho_i \to \rho_f$}

Like in Sec.~\ref{sec:closed-mixed} we choose $X~=~\cvec(\rho)$ and
minimize $D(X_f, X_n) = D(\rho_f, \rho_n)$.
Likewise, we have
$|X|^2 = P(\rho)$, but general Markovian propagation does not
preserve this:
\be
|X_n|^2 = \tr\left(X_i^\dagger \left(\prod_{k=1}^{n} P_k\right)^\dagger \left(\prod_{k=1}^{n} P_k\right) X_i\right).
\ee
If $G_k$ is normal $\eq \quad [G_k, G^\dagger_k] = 0$, we have
\be
P_k^\dagger P_k
= \exp(-\tau_k G^\dagger_k) \exp(-\tau_k G_k)
= \exp(-\tau_k (G_k^\dagger + G_k)).
\ee
If all the generators $G_k$ are antihermitian, this reduces to $\I$, and thus
$|X_n|^2 = \tr(X_i^\dagger X_i) = |X_i|^2$.

Usually this is not the case, (X1) is not satisfied and fidelity does not uniquely define the distance.
%If it does, maximize it:
%\be
%|f(X_f, X_n)| \le \frac{|X_i|}{|X_f|}.
%\ee
%Otherwise,
Hence we directly minimize~$D(X_f, X_n)$.



\subsubsection{General quantum map}

General quantum map operator $X = F$ in Liouville space,
minimize operator distance~$D(X_f, X_n)$.

A unitary target map,
$X_f = \hat{V}$, for example, gives $|X_f|^2 = N^2$.
However, as shown above, the norm $|X_n|^2$ of the propagated operator
is not necessarily constant and we must again minimize the full distance function~$D(X_f, X_n)$.




\subsection{Closed system and environment S+E}

In this case we have a system~$S$ coherently coupled to an
environment~$E$, but are only interested in the state of the system.
The $S$+$E$ propagators~$P_k$ are unitary.

partial distance measures~\cite{kosut_2006}.

\subsection{Open system and environment coupled to a bath S+E+B}

TODO




\section{Package contents}

\subsection{Test suite}
\emph{test\_suite.m} implements all the test optimization problems used in~\cite{machnes_2011}.
Given the number of the problem as input, it initializes a Dynamo
instance with the physics and optimization parameters of that particular problem.

\appendix
\section{Hilbert-Schmidt inner product}

We use the Hilbert-Schmidt inner product for both vectors and matrices:
\be
\inprod{X}{Y} := \tr\left(X^\dagger Y\right).
\ee
It induces the Frobenius norm:
\be
|X| := \sqrt{\inprod{X}{X}} = \sqrt{\tr\left(X^\dagger X\right)}.
\ee

\section{$\cvec$ mapping}
\label{sec:vec}

The $\cvec$ function maps Hilbert space operators (square matrices) to
Liouville space vectors by stacking the columns of the matrix in order
from left to right into a column vector. This mapping is clearly
invertible, and we also have
\be
\cvec(A \rho B) = (B^T \otimes A) \cvec(\rho)
\ee
for any Hilbert space operators~$A, B$.
Consequently, the Liouville space equivalent~$\hat{U}$ for a unitary Hilbert space
propagator~$U$ is
\be
\label{eq:L-unitary}
\hat{U} := U^* \otimes U,
\ee
since
\be
\cvec(U \rho U^\dagger) = (U^* \otimes U) \cvec(\rho) = \hat{U} \cvec(\rho).
\ee
We then have for any~$A$,~$B$
\be
\label{eq:hat-product}
\inprod{\hat{A}}{\hat{B}}
= \tr((A^* \otimes A)^\dagger (B^* \otimes B))
= \tr((A^T B^*) \otimes (A^\dagger B))
%= \tr((A^\dagger B)^* \otimes (A^\dagger B))
= (\tr(A^\dagger B))^* \: \tr(A^\dagger B)
= |\inprod{A}{B}|^2.
\ee

The following property is also easy to verify:
\be
\inprod{\cvec(\rho)}{\cvec(\sigma)} = \cvec(\rho)^\dagger \cvec(\sigma)
= \tr(\rho^\dagger \sigma) = \inprod{\rho}{\sigma}.
\ee





\bibliography{dynamo}
\end{document}
